\documentclass{beamer}
\usetheme{metropolis} % Use the metropolis theme



% Add tikz and pgfplots packages
\usepackage{tikz, pgfplots}
\usetikzlibrary{positioning}

% For clicking references
\usepackage{hyperref}

% For better referencing
\usepackage{cleveref}

\usepackage{graphicx}

% Define custom pastel colors
\definecolor{pastelRed}{RGB}{255, 105, 97}   % A soft pastel red
\definecolor{pastelBlue}{RGB}{119, 158, 203} % A muted pastel blue
\definecolor{pastelYellow}{RGB}{255, 223, 0} % A gentle pastel yellow
\definecolor{lightGray}{RGB}{211, 211, 211}  % A light gray for subtitles and less emphasized text

% Apply the custom colors
\setbeamercolor{palette primary}{bg=black, fg=white}
\setbeamercolor{palette secondary}{bg=lightGray, fg=black}
\setbeamercolor{palette tertiary}{bg=black, fg=white}
\setbeamercolor{titlelike}{parent=palette primary, fg=black}
\setbeamercolor{subtitle}{fg=lightGray}
\setbeamercolor{structure}{fg=black} % For itemize, enumerate, etc

% Change color of normal text
\setbeamercolor{normal text}{fg=black, bg=white}

% Set the color of the table of contents
\setbeamercolor{section in toc}{fg=black} % Section titles in TOC
\setbeamercolor{subsection in toc}{fg=black} % Subsection titles in TOC

% Set block colors
\setbeamercolor{block title}{use=structure,fg=white,bg=pastelRed}
\setbeamercolor{block body}{fg=black,bg=white}



% Title Page Info
\title{Discrete Probability, Random Variables, and Bounds}
\subtitle{Spørgsmål 3 fra Exam Questions}
\author{Kevin Vinther}
\date{\today}

\begin{document}

% Title Page
\begin{frame}
    \titlepage
\end{frame}

% Table of Contents
\begin{frame}[allowframebreaks]
    \frametitle{Table of Contents}
    \tableofcontents
\end{frame}

\section{An Introduction to Discrete Probability}

\begin{frame}[allowframebreaks]
  \frametitle{Finite Probability}
  \begin{itemize}
    \item \textbf{Experiment}: En procedure som giver et resultat ud af et givet sæt a mulige resultater.
    \item \textbf{Sample Space}: Sættet af alle mulige resultater.
    \item \textbf{Event}: Subset af sample space.
  \end{itemize} 
\end{frame}

\begin{frame}
  \frametitle{Definition 1}
  \begin{definition}
If $S$ is a finite nonempty sample space of equally likely outcomes, and $E$ is an event, that is, a subset of $S$, then the \textit{probability} of $E$ is $p(E) = \frac{|E|}{|S|}$.
\end{definition}
\begin{itemize}
\item \textit{Implikationer:}
\item Givet, at $E$ er en event, og $S$ sample space:
  \item $0 \leq |E| \leq |S|$, fordi $E \subseteq S$. Dermed $0 \leq p(E) = |E|/|S| \leq 1$.
\end{itemize}
\end{frame}

\begin{frame}
  \frametitle{Example 1}
  \begin{itemize}
  \item<1-> En boks indeholder fire blå bolde og fem røde. Hvad er sandsynligheden at en bold der er valgt tilfældigt er blå?
    \item<2-> $\frac{4}{9}$
  \end{itemize}
\end{frame}

\begin{frame}
  \frametitle{Example 2}
 \begin{itemize}
 \item<1-> Hvad er sandsynligheden for at, når 2 terninger bliver rullet, at summen af de to tal på terningen er 7?
 \item<1-> $6^2 = 36$ mulige udfald. (Af product rule).
 \item<2-> Der er 6 udfald der ender i 7: (1,6), (2,5), (3,4), (4,3), (5,2), og (6,1)
   \item<3-> Dermed: $\frac{6}{36} = \frac{1}{6}$
 \end{itemize} 
\end{frame}

\begin{frame}
  \frametitle{Eksempel 5}
\begin{itemize}
\item<1-> Hvad er sandsynligheden for at en hånd i poker har 4 kort af én slags (samme tal)? (Hint: Product rule)
\item<2-> Antallet af hænder med 5 kort med fire af én slags er produktet af antallet måder man kan vælge én slags på, antallet af måder du kan vælge disse 4 af slagsen ud af de fire i dækket, og antallet af måder du kan vælge det femte kort. Det er:
\item<3-> $\binom{13}{1}\binom{4}{4}\binom{48}{1}$. Næste: Hvor mange forskellige hænder af 5 kort er der? 
\item<4-> $\binom{52}{5}$. Næste: Find sandsynligheden
  \item<5-> $\frac{\binom{13}{1}\binom{4}{4}\binom{48}{1}}{\binom{52}{5}}= \frac{13 \cdot 1 \cdot 48}{2598960} \approx 0.00024$
\end{itemize}  
\end{frame}

\begin{frame}
  \frametitle{Eksempel 7 i}
 \begin{itemize}
 \item<1-> Hvad er sandsynligheden at tallene 11, 4, 17, 39 og 23 bliver taget i den orden fra en spand med 50 bolde med tallene 1..50, hvis (a) bolden der er valgt ikke bliver lagt tilbage i spanden før den næste bold blive valgt og (b) bolden bliver lagt tilbage før den næste bold bliver valgt?
 \item<1-> Lad os starte med spørgsmål (a) (hint: Hvor mange måder kan man vælge boldene på?):
 \item<2-> Man kan vælge 5 bolde på $50 \cdot 49 \cdot 48 \cdot 47 \cdot 46 = 254251200$ måder. Hvad er sandsynligheden så for at vælge præcis den orden der er beskrevet tidligere?
 \item<3-> $\frac{1}{254251200}$.
   \item<3-> Bogen kalder dette \textbf{sampling without replacement}.
 \end{itemize} 
\end{frame}

\begin{frame}
  \frametitle{Eksempel 7 ii}
 \begin{itemize}
 \item<1-> Hvad med (b)? Hvor mange måder kan man vælge 5 bolde på så?
 \item<2-> $50^5$. Husk at antallet af permutationer med gentagelser er $n^r$, og her er $n = 50, r = 5$.
 \item<3-> Måden man finder sandsynligheden på har ikke ændret sig her.
 \item<3-> $\frac{1}{312500000}$
 \item<3-> Bogen kalder dette \textbf{sampling with replacement}.
   \item<3-> Hvad forskellen er mellem det og permutations with repetition er et godt spørgsmål.
 \end{itemize} 
\end{frame}

\subsection{Probabilities of Complements and Unions of Events}
\label{subsec:probcompunion}

\begin{frame}
  \frametitle{Theorem 1}
  Tid til første theorem!
  \begin{theorem}[Theorem 1]
    Let $E$ be an event in a sample space $S$. The probability of the event $\overline{E} = S - E$, the complementary event of $E$ is given by $$P(\overline{E}) = 1 - p(E)$$
  \end{theorem}
\end{frame}

\begin{frame}
  \frametitle{Theorem 1 Bevis}
  \begin{proof}[Theorem 1 Bevis]
For at finde sandsynligheden af begivenheden $\overline{E} = S - E$, notér at $|\overline{E}| = |S| - |E|$. Dermed: $$p(\overline{E}) = \frac{|S|-|E|}{|S|}=1 - \frac{|E|}{|S|}=1-p(E)$$
  \end{proof}
\end{frame}

\begin{frame}
  \frametitle{Eksempel 8}
  \begin{itemize}
  \item<1-> En sekvens af 10 bits er tilfældigt genereret. Hvad er sandsynligheden for at mindst en af disse bits er 0?
  \item<1-> Hint: Brug complement.
  \item<2-> Lad $E$ være tilfældet at mindst en af de 10 bits er 0. Så er $\overline{E}$ sandsynligheden for at alle bitsne er 1. 
  \item<3-> $P(E) = 1-p(\overline{E})=1- \frac{|\overline{E}|}{|S|}=1- \frac{1}{2^{10}} = 1 - \frac{1}{1024} = \frac{1023}{1024}$
  \end{itemize}
\end{frame}

\begin{frame}
  \frametitle{Theorem 2}
  \begin{theorem}
Let $E_{1}$ and $E_{2}$ be events in the sample space $S$. Then $$p(E_{1} \cup E_{2}) = p(E_{1})+P(E_{2})-p(E_{1} \cap E_{2}).$$
  \end{theorem} 
\end{frame}

\begin{frame}
  \frametitle{Theorem 2 Bevis}
  \begin{proof}[Theorem 2 Bevis]
    Vi bruger subtraction rule for sæt (inclusion-exclusion for 2 sæt):
    $$|E_{1} \cup E_{2}| = |E_{1}| + |E_{2}| - |E_{1} \cap E_{2}|$$
    Dermed:
    \begin{equation*}
      \begin{split}
        p(E_{1} \cup E_{2}) &= \frac{|E_{1} \cup E_{2}|}{|S|}\\
                         &= \frac{|E_{1}|+|E_{2}-|E_{1}\cap E_{2}|}{|S|}\\
                         &= \frac{|E_{1}|}{|S|} + \frac{|E_{2}|}{|S|}- \frac{|E_{1} \cap E_{2}|}{|S|}\\
        &= p(E_{1})+p(E_{2})-p(E_{1} \cap E_{2})
      \end{split}
    \end{equation*}
  \end{proof}
\end{frame}

\begin{frame}
  \frametitle{Eksempel 9}
 \begin{itemize}
 \item<1-> Hvad er sandsynligheden for at et positivt heltal valgt tilfældligt fra et sæt af positive heltal der ikke overgår 100 er deleligt med 2 eller 5?
 \item<1-> Hint: Definér $E_{1}$ og $E_{2}$ som værende tal der er delelige med hhv. 2 og 5.
 \item<2-> $E_{1}$ = tal der er delelige med 2, $E_{2}$ = tal der er delelige med 5.
 \item<3-> $|E_{1}| = 50$, $|E_{2}| = 20$, $|E_{1} \cup E_{2}| = 10$.
 \item<3-> Vi ved at $p(E_{1}\cupE_{2})=p(E_{1})+P(E_{2})-p(E_{1} \cap E_{2})$. Så: $= \frac{50}{100}+ \frac{20}{100}- \frac{10}{100} = \frac{3}{5}$
     
 \end{itemize} 
\end{frame}

\subsection{Probabilistic Reasoning}
\label{subsec:probabilistic-reasoning}

\begin{frame}
  \frametitle{Probabilistic Reasonining}
 \begin{itemize}
 \item Probabilistic Reasoning er at bruge sandsynlighedsregning til at ræsonnere sig frem til resultater. Et eksempel på hvor man kan gøre dette, er Monty Hall Three-Door Puzzle, som vi kommer til nu!
 \end{itemize} 
\end{frame}

\begin{frame}
  \frametitle{Eksempel 10: The Monty Hall Three-Door Puzzle}
  \begin{itemize}
  \item Antag at du er en spiller i et game show.
  \item Du bliver bedt om at vælge en af tre døre.
  \item Den store pris er bag en af de tre døre, og de to andre døre leder til tab.
  \item Når du vælger en dør, vil værten lade dig enten beholde døren, eller vælge en ny. Du får ikke lov til at se hvad der er i døren hvis du vælger en ny.
  \item Vælger du en ny? 
  \end{itemize}
\end{frame}

\begin{frame}
  \frametitle{Eksempel 10 Løsning}
  
  \begin{itemize}
  \item Ja du gør!
  \item Sandsynligheden for at du har valgt den rigtige dør først er $\frac{1}{3}$
  \item Sandsynligheden for at du har valg \textbf{forkert} er $\frac{2}{3}$. 
  \item Hvis du valgte forkert, åbner game show værten en dør hvor der ingen præmie er bag.
  \item Dermed, kommer du \textbf{altid} til at vinde, hvis dit første valg var forkert, og du så vælger at skifte døre.
  \item Så, ved at skifte døre er chancen for at vinde $\frac{2}{3}$
  \end{itemize}
\end{frame}

\section{Probability Theory}
\label{sec:probability-theortheory}

\subsection{Introduction}
\label{subsec:probability-theory-introduction}





\end{document}