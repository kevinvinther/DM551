\documentclass{beamer}
\usetheme{metropolis} % Use the metropolis theme




% Add tikz and pgfplots packages
\usepackage{tikz, pgfplots}
\usetikzlibrary{positioning}

% For clicking references
\usepackage{hyperref}

% For better referencing
\usepackage{cleveref}

\usepackage{graphicx}


\usepackage{amsmath}
\usepackage{algorithm}
\usepackage{algpseudocode}

% Define custom pastel colors
\definecolor{pastelRed}{RGB}{255, 105, 97}   % A soft pastel red
\definecolor{pastelBlue}{RGB}{119, 158, 203} % A muted pastel blue
\definecolor{pastelYellow}{RGB}{255, 223, 0} % A gentle pastel yellow
\definecolor{lightGray}{RGB}{211, 211, 211}  % A light gray for subtitles and less emphasized text

% Apply the custom colors
\setbeamercolor{palette primary}{bg=black, fg=white}
\setbeamercolor{palette secondary}{bg=lightGray, fg=black}
\setbeamercolor{palette tertiary}{bg=black, fg=white}
\setbeamercolor{titlelike}{parent=palette primary, fg=black}
\setbeamercolor{subtitle}{fg=lightGray}
\setbeamercolor{structure}{fg=black} % For itemize, enumerate, etc

% Change color of normal text
\setbeamercolor{normal text}{fg=black, bg=white}

% Set the color of the table of contents
\setbeamercolor{section in toc}{fg=black} % Section titles in TOC
\setbeamercolor{subsection in toc}{fg=black} % Subsection titles in TOC

% Set block colors
\setbeamercolor{block title}{use=structure,fg=white,bg=pastelRed}
\setbeamercolor{block body}{fg=black,bg=white}



% Title Page Info
\title{Randomized Algorithms}
\subtitle{Spørgsmål 4 fra Exam Questions}
\author{Kevin Vinther}
\date{\today}

\begin{document}

% Title Page
\begin{frame}
  \titlepage
\end{frame}

% Table of Contents
\begin{frame}[allowframebreaks]
  \frametitle{Table of Contents}
  \tableofcontents
\end{frame}

\section{Randomly Permuting Arrays}
\label{sec:cormen5}


\begin{frame}
  \frametitle{Randomly Permuting Arrays}
  \begin{itemize}
  \item Givet et array, vil vi gerne have en måde hvorpå vi kan permutere det tilfdæligt. 
  \item Der er flere måder at gøre dette på.
  \item En af måderne:
    \begin{itemize}
    \item Giv hvert element i array A[i] en prioritet P[i], og sortér after prioriteterne.
    \end{itemize}
  \end{itemize}
\end{frame}

\begin{frame}
  \frametitle{Permute-By-Sorting}
  
  \begin{algorithmic}[1]
      \State $n \gets A.\text{length}$
      \State Let $P[1..n]$ be a new array
      \For{$i \gets 1 \text{ to } n$}
      \State $P[i] \gets \text{RANDOM}(1, n^3)$
      \EndFor
      \State sort $A$, using $P$ as sort keys
    \end{algorithmic}
\end{frame}

\begin{frame}
  \frametitle{Lemma 5.4}
  \begin{lemma}[Lemma 5.4]
  Procedure \texttt{Permute-By-Sorting} produces a uniform random permutation of the input, assuming that all priorities are distinct
  \end{lemma}
\end{frame}

\begin{frame}[allowframebreaks]
  \frametitle{Lemma 5.4 Bevis}
 Vi starter med at overveje den præcise permutation hvor elementet $A[i]$ får den $i$'e laveste prioritet. Vi viser at denne permutation sker med den præcise sandsynlighed $1/n!, for i =, 1, 2, \ldots, n$. Lad $E_{i}$ være hændelsen at elementet $A_{i}$ får den $i'e$ mindste prioritet. Vi ønsker at udregne sandsynligheden for alle $i$.

   \begin{equation*}
P(E_{1} \cap E_{2} \cap \cdots \cap E_{n-1} \cap E_{n})
   \end{equation*}

   Denne sandsynlighed er lig med (iflg. opgave C.2-5):

   \begin{equation*}
     P(E_{1}) \cdot P(E_{2}|E_{1}) 
     \cdots P(E_{i}| E_{i-1} \cap E_{i-2} \cap \cdots \cap E_{1}) \cdots P(E_{n}|E_{n-1} \cap \cdots \cap E_{1}) 
   \end{equation*}

   Vi har at $p(E_{1}) = 1/n = 1/n = 1/n = 1/n$ fordi det er sandsynligheden at en prioritet valgt tildældigt ud af et sæt af $n$ er den mindste sandsynlighed.

   Som det næste observerer vi at $P(E_2|E_{1}) = \frac{1}{n-1}$ fordi, givet at elementet $A[1]$ har den mindste prioritet, har hver af de resterende $n-1$ elementer en lige chance for at have den anden mindste prioritet. Generelt:

   \begin{equation*}
     p(E_{i} | E_{i-1} \cap E_{i-2} \cap \cdots \cap E_{1}) = \frac{1}{(n-i+1)}
   \end{equation*}

   Dermed har vi:

   \begin{equation*}
     \begin{split}
     p(E_{1} \cap E_{2} \cap E_{3} \cap \cdots \cap E_{n-1} \cap E_{n}) &=\\
     \left( \frac{1}{n} \right) \left( \frac{1}{n-1} \right) &\cdots \left( \frac{1}{2} \right) \left( \frac{1}{1} \right) \\
    &= \frac{1}{n!}\\
     \end{split}
   \end{equation*}
\end{frame}

\begin{frame}
  \frametitle{Randomize-in-Place}

  \begin{itemize}
  \item Vi kan gøre det endnu bedre. Permute-By-Sorting var $\Theta (n \lg n)$. Randomize-In-Place er $O(n)$
  \end{itemize}
\end{frame}

\begin{frame}
  \frametitle{Randomize-in-Place}
  \begin{algorithmic}[1]
      \State $n \gets A.\text{length}$
      \For{$i \gets 1 \text{ to } n$}
      \State $\text{swap }A[i]\text{ with }A[Random(i,n)] $
      \EndFor
    \end{algorithmic}
\end{frame}

\begin{frame}
  \frametitle{Lemma 5.5}

  \begin{lemma}[Lemma 5.5]
    Procedure \texttt{Randomize-in-Place} computes a uniform random permutation.
  \end{lemma}
\end{frame}

\begin{frame}[allowframebreaks]
  \frametitle{Lemma 5.5 Bevis}

  Vi bruger den følgende loop invariant\footnote{En invariant skal være sand under hele beviset}:
 Lige før den $i'e$  iteration af \textbf{for} loopet (linjerne 2-3), for hver mulig $(i-1)$-permutation af de $n$ elementer; indeholder subarrayen $A[1..i-1]$ denne $(i-1)$-permutation med sandsynligheden $(n-i+1)!/n!$

 (Altså, sandsynligheden for at subarrayen indtil videre er en præcis subarray skal gerne være $(n-i+1)!/n!$ (for ellers er den ikke uniformly random.))

 Vi skal så vise at invarianten er sand:
 \begin{itemize}
 \item Før den første loop
 \item I hvert loop efterfølgende
 \item Når loopet er færdigt
 \end{itemize}
\end{frame}

\end{document}
%%% Local Variables:
%%% mode: latex
%%% TeX-engine: xetex
%%% TeX-command-extra-options: "-shell-escape"
%%% End: